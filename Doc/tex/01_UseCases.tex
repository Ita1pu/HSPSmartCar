\chapter{Use Cases}
\label{sec:useCases}
Die Daten aus der OBD II Schnittstelle eines Autos, vereint mit dem GPS-Sensor und den anderen Sensoren des Dongles, bieten für eine Auswertung nahe zu unbegrenzte Möglichkeiten. Aus den Analysen entstehen Anwendungsfälle, für die sich das SmartCar System verwenden lässt. Das nachfolgende Kapitel beinhaltet eine Auswahl an Use Cases auf dessen Grundlage das Design und die Implementierung der einzelnen Komponenten entstanden ist.
\section{Aufzeichnen von Routen}
Ein Anwendungsbeispiel für SmartCar ist das Aufzeichnen von Routen während der der Fahrt. Der Dongle soll, nach dem Einstecken in die OBD-Schnittstelle, mithilfe des zusätzlichen GPS Moduls Fahrten aufzeichnen und auf einer SD-Karte speichern. Für jede Fahrt soll hier eine separate Datei erstellt werden. Zusätzlich zu den GPS-Koordinaten legt der Dongle noch ausgewählte Daten aus der OBD Schnittstelle auf der SD-Karte ab. Diese Aufzeichnung erfolgt so, dass eine Darstellung und Abspeicherung in einem externen System(Backend) möglich ist und dort auch einzelne Fahrten unterscheidbar sind.
\section{Analyse der aufgezeichneten Daten}
Aus dem Abspeichern der oben genannten Informationen ergibt sich auch der zweite Anwendungsfall. Die Daten, die der Dongle abspeichert soll ein Backend anschaulich darstellen und Auswertungen über das Erfasste anfertigen. Solche Darstellungen sind z.B. Heat-maps, auf denen Daten wie Geschwindigkeit, Motordrehzahl oder Spritverbrauch für jede Fahrt zu sehen sind. So ist es möglich zu erkennen auf welchen Strecken wann welche Daten erfasst wurden. Ein Nutzer kann z.B. dann einsehen, wann er auf welchen Strecken wie viel Treibstoff verbraucht hat und dadurch seine Fahrgewohnheiten optimieren.
\section{Live Daten Anzeige von OBD Daten auf dem Smartphone}
Die vom Dongle erfassten Daten sind aber nicht nur für eine spätere Analyse interessant, sondern liefern auch während der Fahrt wertvolle Informationen. Aus diesem Grund ist ein weiterer Einsatzzweck das Darstellen von bestimmten Live-Daten auf einer Smartphone App. Die Kommunikation zwischen App und Smartphone erfolgt hier über Bluetooth Low Energy(BLE). Solche Live-Daten sind z.B. der aktuelle Verbrauch des Autos oder die derzeitige Geschwindigkeit. Das ist vor allem für ältere Autos interessant, da sie Daten wie Spritverbrauch meistens selbst noch nicht anzeigen. Aber auch Fahrer von neueren Autos können so die Ausgaben ihrer Autos mit denen der App vergleichen und etwaige Fehler im Fahrzeug erkennen
 