\section{App}

\subsection{Grundgerüst}
\label{sec:appGrundgerust}

\begin{itemize}
\item script.ts
\item variables.ts (zusammenspiel zwischen Typescript und Scss), das variables.scss generiert wird (mit beispiel)
\item embedden von resources (images in scss) (wie wird das gemacht, vllt das script beschreiben)
(evtl. mit neuem VSC-TS scripten) 
\end{itemize}

\subsection{Bluetooth}
\label{sec:appBluetooth}

\begin{itemize}
\item zugriff auf Bluetooth (abstraktion über cordova framework)
\end{itemize}

\subsection{GPS}
\label{sec:appGPS}

\begin{itemize}
\item todo
\end{itemize}

\subsection{Zugriff auf Identity Server}

\begin{itemize}
\item Ajax request zum bekommen des Tokens (über den kann dann später für die schnittstelle zum Backend authentifiziert werden wird im http header übergeben)
\item literatur https://tools.ietf.org/html/rfc6750
\end{itemize}

\subsection{Konfiguration}
\label{sec:appKonfiguration}
\begin{itemize}
\item Einstellungsmöglichkeiten
\item Verwaltung, speicherung, Initialwerte von einstellungen (Sprache von system sprache geladen)
\item sowohl die vom User als auch die fest im code (settings.ts, store.ts)
\end{itemize}

\subsection{Mehrsprachigkeit}
\label{sec:appMehrsprachigkeit}

\begin{itemize}
\item strings.ts, funktion "S"
\end{itemize}

\subsection{Sichten}

\subsubsection{Konfiguration}
\label{sec:appSichtKonfiguration}

\begin{itemize}
\item jede einzelen komponente durchgehen
\item zusammenspiele zu anderen komponenten über events und store usw..., .. Einstellungsmöglichkeiten usw)
\item bild von suchen von bluetooth devices
\item bild von connect to backend
\item bild von Logging
\item usw..
\end{itemize}

\subsubsection{Anzeige}
\label{sec:appSichtAnzeige}

\begin{itemize}
\item Was gibt es alles für anzeigen screenshots und kurze erklärung
\item welche daten werden benötigt, wie bekommt man die (handler = events von anderen komponenten abonieren) 
\end{itemize}

\subsection{Weitere Anzeigeelemente}
\label{sec:appAnzeigeelemente}

\begin{itemize}
\item display features
\item dialog
\item ViewCircles
\item ErrorPanel
\item SwipeHelp
\end{itemize}

\subsection{Test}
\label{sec:appTest}

\begin{itemize}
\item vllt ins testkapitel
\item Wie debug system funktioniert chrome://inspect/\#devices (evtl bild)
\item Integrationstest zu Dongle (Test außerhalb und vorallem im Auto)
\item Integrationstest zu Identity Server Get Token für Authentifizierung vom Identity Server
\end{itemize}

\subsection{Ausblick}
\label{sec:appAusblick}

\begin{itemize}
\item vllt ins ausblickkapitel
\item Upload über die App
\item beliebig viele weitere Views (sehr einfach umzusetzen über das View system)
\item Flashen von neuer Donglesoftware
\end{itemize}
  

 
 
 
 
 
 
 
 