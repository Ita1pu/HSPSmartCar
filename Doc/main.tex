\documentclass[numbers=noenddot, a4paper, 12pt, headsepline, footsepline]{scrreprt}

% Deutsche Umlaute
\usepackage[utf8]{inputenc}

% deutsche Silbentrennung
\usepackage[ngerman]{babel}

%anführungszeichen
\usepackage[autostyle=true,german=quotes]{csquotes}

% Einbinden von Bildern
\usepackage{graphicx}

% Mathematische Pakete
\usepackage{amsmath}
\usepackage{amssymb}
\usepackage{amsfonts}
\usepackage{amsthm}
\usepackage{latexsym}
\usepackage{morefloats}
\usepackage{geometry} 


\usepackage{standalone}

% better looking tables
\usepackage{ifthen}
\usepackage{booktabs}
\usepackage{multirow}


% ausgabe von quelltext
\usepackage{textcomp}
\usepackage{listings}

%einbinden von pdf seiten
\usepackage{pdfpages}

\lstset{
	%	backgroundcolor=\color{lbcolor},
	tabsize=4,    
	%	rulecolor=,
	language=[GNU]C++,
	basicstyle=\scriptsize,
	upquote=true,
	%aboveskip={1.5\baselineskip},
	columns=fixed,
	showstringspaces=false,
	extendedchars=false,
	breaklines=true,
	%prebreak = \raisebox{0ex}[0ex][0ex]{\ensuremath{\hookleftarrow}},
	%frame=single,
	numbers=none,
	showtabs=false,
	showspaces=false,
	showstringspaces=false,
	identifierstyle=\ttfamily,
	language=C++,
	basicstyle=\ttfamily,
	keywordstyle=\color{blue}\ttfamily,
	stringstyle=\color{red}\ttfamily,
	commentstyle=\color[rgb]{0,.7,0}\ttfamily,
	morecomment=[l][\color{magenta}]{\#},
	%\lstdefinestyle{C++}{language=C++,style=numbers}’,
}
\lstset{
	basicstyle=\renewcommand{\baselinestretch}{.9}\ttfamily,
	xleftmargin=.5cm,
	language=C++,
	numbers=left,
	stringstyle=\color{black}\ttfamily,
	keywordstyle=\color{black}\ttfamily,
	numberstyle=\tiny}


% coolere referenzen
\usepackage[german]{fancyref}

%literatur mittels biber
\usepackage[backend=biber,
			style=alphabetic
			%citestyle=alphabetic 
			]{biblatex}

%\ExecuteBibliographyOptions{
%	sorting=nyt, %Sortierung Autor, Titel, Jahr
%	bibwarn=true, %Probleme mit den Daten, die Backend betreffen anzeigen
%}
\addbibresource{literatur.bib}

%querformat seiten
\usepackage{lscape}

%überschrift des Kapitel auf jeder Seite
\pagestyle{headings}

%lange tabellen mit seitenumbruch
\usepackage{longtable}

%abkürzungsverzeichnis
\usepackage{acronym}

%um links im pdf zu haben mit denen man hin und her springen kann
\usepackage{hyperref}

%%text farbig hinterlegen
%\usepackage{color}

%%um text durch, unter, über... streichen zu können
%\usepackage{ulem}

%um in tabellen zellen diagonal teilen zu können
%\usepackage{diagbox}


\begin{document}

%einbinden titelblatt
\title{Smart Car} 
\subtitle{Hauptseminar Projektstudium}
\author{\\
	\begin{tabular}{|c|c|c|}
		\hline 
		Knorr Thomas 		 & xyzk 	& HSP1 \\ 
		\hline 
		Lackner Andreas 	 & 3120204 	& HSP2 \\ 
		\hline 
		Schleinkofer Michael & xyzk 	& HSP2 \\ 
		\hline 
		Welker Franz 		 & 3119754  & HSP2 \\ 
		\hline 
		Wiche Fabian 		 & xyzk		& HSP1 \\ 
		\hline 
	\end{tabular} 
	}	

\date{\today}
%\clearpairofpagestyles

\makeatletter
\begin{titlepage}
	
	\begin{center}
		\includegraphics[width=10cm]{./img/OTHLogo.jpg}\\
		\vspace{4cm}
		{\huge\bfseries\@title\unskip\strut\par}\paragraph{}
		{\Large\bfseries\@subtitle\unskip\strut\par}\paragraph{}
		An der\\
		Ostbayerischen Technischen Hochschule Regensburg\\
		Fakultät Informatik/Mathematik
		\paragraph{}
		Betreut durch Prof. Dr. Klaus Volbert\\
		\vspace{\fill}
		Vorgelegt von: \vspace{0,3cm}
		\@author\\
		\vspace{0,3cm}
		Datum: \@date
	\end{center}
\end{titlepage}
\makeatother



%arabische zahlen für seiten
\pagenumbering{arabic}

% Inhaltsverzeichnis anzeigen
%\parskip am schluss anpassen, so dass ein zweiseitiges inhaltsverzeichnis gut umgebrochen wird
{\parskip=+5mm \tableofcontents}

\include{./TeX_files/introduction}
\include{./TeX_files/chapter01}% Warum haben wir es getan
\chapter{Implementierung}

Wie wurde es umgesetzt, wie waren die Ideen (also hier z.b. auf die Gesamte Architektur eingehen, spezifische der einzelnen Bereiche in den unter Kapiteln)

\section{App}

Was kann die App, und besonderes erklären

  

 
 
 
 
 
 
 
 
\input{./TeX_files/chapter02Backend}
\section{Dongle}

Was kann der Dongle, und besonderes erklären

 
 
 
 
 
 
 
 

  

 
 
 
 
 
 
 
 % Wie haben wir es getan + aufgetretene Probleme
\include{./TeX_files/chapter03}% Was haben wir getestet
\chapter{Ausblick}

Was kann man jetzt mit unserem Zeug als Grundlage anstellen, was haben wir uns überlegt aber konnten es nicht mehr umsetzen.

  

 
 
 
 
 
 
 
 % Was kann man jetzt mit unserer Grundlage weiterhin machen






\newpage
\listoffigures
\lstlistoflistings
\listoftables
% Muss von Hand sortiert werden!!

%anlegen:
% \acro{}[]{}
% \acro{}{}

%verwenden
%\ac{x} dann wird x als verlinkung genommen  
%\acp{x} wird auch x als verlinkung genommen, aber mit einem s am ende für mehrzahl

%\newpage
\section*{Abkürzungsverzeichnis} %Abkürzungsverzeichnis
\markboth{Abkürzungsverzeichnis}{}
\begin{acronym}[ABCDEFGHIJK] 			% längste Abkürzung steht in eckigen Klammern
	\acro{HKIWS}{Hier könnte ihre Werbung stehen}

	
	
	
	
	
	
	
	
	
	
	
	
	
	
	
	
\end{acronym}
\printbibliography

\newpage
\appendix










\end{document}