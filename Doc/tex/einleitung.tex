\chapter{Einleitung}
%Kurze Einleitung, warum haben wir uns das Thema ausgesucht (nicht lang oder so)...
%Gibt z.B. Pace aber braucht man immer Handy, wollten es anders machen...

Ein zu beobachtender Trend ist die zunehmende Digitalisierung verschiedenster Lebensbereiche mit dem Ziel, die Effizient und Bequemlichkeit des Alltags zu steigern. Die Grundvoraussetzung für die Funktionalität bestehender und die Entwicklung neuer Lösungen ist meist eine Datenquelle welche die angebotenen Dienste mit Informationen versorgt. 

Eine Industrie die dieser Trend betrifft, ist die Automobilbranche. Moderne Fahrzeuge sammeln durch die verbauten Sensoren eine Vielzahl von Daten die sowohl den Betrieb des Fahrzeugs selbst, sowie den diverser Sicherheits- und Komfortfunktionen ermöglichen. Dabei nimmt ebenfalls die Zahl der Informationen zu, die mit dem Fahrer bzw. der Umwelt geteilt werden. Dies umfasst beispielsweise den Kraftstoffverbrauch, verbleibende Reichweite oder die Motorlast sowie Positions- und Telemtriedaten die unter anderem zukünftig mit anderen Fahrzeugen oder Serviceanbietern geteilt werden. Neben den Vorteilen für andere Verkehrsteilnehmer, durch verbesserte Unfallvermeidung oder Stauerkennung, sind die Hauptprofiteure die Fahrzeugführer bzw. Besitzer die dadurch eine verbesserte Übersicht über ihr Fahrzeug und Fahrverhalten bekommen.

Von dieser Entwicklung ausgeschlossen sind vor allem Besitzer älterer Fahrzeuge, die oft nur eine begrenzte Zahl von Informationen über die Kombiinstrumente teilen und keine Schnittstelle besitzen, um automatisiert Daten zu extrahieren. 

\section{Verfügbare Lösungen}
\label{sec:solutions}
Um auch Fahrzeuge älteren Baujahres oder eingeschränkter Ausstattung mit den Vorteilen moderner Analyse und Statistik Applikationen auszustatten, sind am Markt verschiedene Lösungen verfügbar:

\begin{itemize}
\item Smartphone Applikationen \\
Hier finden sich in den App Stores verschiedene Anwendungen die Services wie das automatisierte führen von Fahrtenbüchern \footnote{\url{https://www.mycartracks.com/}} oder Navigation gepaart mit Echtzeitmeldungen über Stau oder andere Hindernisse anbieten\footnote{\url{https://www.waze.com/de/}}. All diese Applikationen verwenden für ihren Betrieb entweder nur Positionsdaten durch das GPS Modul des Smartphones oder manuelle Eingaben der Fahrer. Die Nachteile bestehen darin, dass sowohl viel Interaktion mit der App erforderlich ist (Start und Stopp bei Fahrtantritt und -ende) als auch keine Fahrzeuginternen Daten zur Verfügung stehen, die vor allem bei Fahrzeugen ohne Boardcomputer einen bedeutenden Mehrwert darstellen.
\item Pace\footnote{\url{https://www.pace.car/de}}\\
Das Produkt der Firma Pace löst das Problem der fehlenden Fahrzeugschnittstelle über einen Adapter der mittels OBDII Zugriff auf Fahrzeugdaten ermöglicht. Gepaart mit einer App können so Services wie ein Performance Monitor, Fehleranalyse, Spritspartraining oder ein elektronisches Fahrtenbuch angeboten werden. Der ständige Betrieb der App ist dabei zwingend erforderlich, da der eingesetzte Adapter weder über internen Speicher noch über ein eigenes GPS Modul verfügt, um eigenständig alle relevanten Daten aufzuzeichnen. Auch hier ist dadurch die ständige manuelle Interaktion des Fahrers notwendig.
\item Mojio\footnote{\url{https://www.moj.io/}}\\
Im Unterschied zu den zuvor vorgestellten Ansätzen ist das Ziel von Mojio nicht nur ein Gadget für Fahrzeugbesitzer anzubieten, sondern eine Plattform für Fahrzeugdaten zu entwickeln. Die Telemetriedaten werden wie auch bei Pace über einen OBDII Adapter erfasst und in das Mojio Backend geladen. Die Firma stellt dabei lediglich die Schnittstellen und die Plattform zur Verfügung, weshalb die Anwendung nicht auf einen speziellen Adapter beschränkt ist. Mit den von vielen Fahrzeugen gesammelten Daten können im Nachgang verschiedene Statistiken und Analysen erstellt oder eigenen Applikationen mit Daten versorgt werden.
\end{itemize}

\section{Zielsetzung}
Ziel des Projekts ist die Entwicklung einer ganzheitlichen Lösung, um Fahrzeuge die nur über eingeschränkte Informationsschnittstellen verfügen nachträglich mit den Vorzügen moderner Visualisierungs- und Statistikapplikationen auszustatten. Dazu werden die Vorteile der in \ref{sec:solutions} beschriebenen Lösungen vereint. Um die Fahrzeugdaten extrahieren zu können, wird ein Programmierbarer OBDII Dongle verwendet, der zusätzlich über einen eigenen GPS Empfänger und Speicher verfügt, um unabhängig von einem Smartphone Daten aufzeichnen zu können. Die zusätzliche App mit deren Hilfe sich verschiedene Fahrzeugparameter in Echtzeit visualisieren lassen, ist damit optional. Für eine statistische Analyse der gesammelten Bewegungs- und Telemetriedaten soll ein Backend zur Verfügung gestellt werden, welches entweder über die auf der Speicherkarte gesicherten Logs oder einen Upload über das Smartphone gespeist wird. Basierend auf dieser Infrastruktur lassen sich in weiterer Folge verschiedene neue Applikationen aufsetzten. \\

Die Gliederung des Berichts beginnt mit einer Aufstellung der Use Cases in \ref{sec:useCases} die von der finalen Lösung umgesetzt werden sollen. Darauf folgt in Kapitel \ref{sec:systemDesign} eine Übersicht über die Systemarchitektur, gefolgt von der Dokumentation der Implementierung in \ref{sec:implementation}, die in die jeweiligen Teilprojekte gegliedert ist. Abgeschlossen wird der Bericht durch einen Test des Gesamtsystems (\ref{sec:test}) sowie einen Ausblick auf mögliche weitere Entwicklungen (\ref{sec:outlook}).









